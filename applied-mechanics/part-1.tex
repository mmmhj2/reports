
\part{航空发动机原理}

\begin{tcolorbox}
    针对常用的航空燃气涡轮发动机,请回答并分析以下问题:
    \begin{enumerate}
        \item 常用的航空燃气涡轮发动机的类型包括哪些,这些发动机在结构上有何异同?
        \item 引发这些结构变化背后的性能影响因素包括哪些,为什么?
    \end{enumerate}
\end{tcolorbox}

\section{第一题}

航空燃气涡轮发动机包括涡轮喷气发动机和复燃加力涡轮喷气发动机、涡轮螺桨发动机、涡轮轴发动机、涡轮风扇发动机、桨扇发动机和变循环和组合循环发动机几个类别。

涡轮喷气式发动机,简称涡喷(Turbojet),利用布雷顿循环向后喷出高速流动的空气来产生推力。
理想布雷顿循环由等熵压缩、等压升温、等熵膨胀和等压放热四个部分组成,而涡轮喷气式发动机从前至后分别具有五个部件:进气道、压气机、燃烧室、涡轮和尾喷管。
压气机中发生压缩,燃烧室中通过加入燃料发生升温,涡轮和尾喷管中发生膨胀,而等压放热则直接在大气中进行,其中涡轮和压气机通过传动轴连接来传递动力。
复燃加力涡喷发动机则在涡轮和尾喷管之间添加一个加力燃烧室(Afterburner),通过额外发生一次等压加热进一步提高循环功。
涡喷发动机能够比较容易地跨过音障,在高速飞行下效率较高,但是在低速情况性能欠佳。

涡轮螺桨发动机,简称涡桨发动机(Turboprop),和涡轮轴发动机(Turboshaft)则不主要通过喷出空气产生推力,而是直接使用涡轮输出的扭矩和轴功。
在固定翼飞行器上,这种轴功率用来驱动螺旋桨,而喷出的气体有时也能补充部分推力,从而形成涡桨发动机;在直升机上,则完全不使用气体的推力,称为涡轮轴发动机。
这样的固定翼和使用活塞发动机的类似,都很难突破音速。
在地面上,如坦克上,使用的涡轮发动机基本都是这种形式。
由于工作原理大致相同,涡桨发动机、涡轴发动机和涡喷发动机的形式也大致相同,一般也由进气道、压气机、燃烧室、涡轮和尾喷管五个部分组成,对涡桨发动机还需要加上螺旋桨。
涡桨发动机在低速情况($\mathrm{Ma} < 0.6$)下表现优秀。

涡轮风扇发动机,简称涡扇发动机(Turbofan),则可视作涡喷发动机和涡桨发动机的结合,以期实现在低速和高速下都有较高的效率。
经过涡扇发动机的空气,一部分经内涵道通过风扇以及包含压气机、燃烧室、涡轮和尾喷管的核心机,实现完整的布雷顿循环;另一部分则通过外涵道,只通过风扇升压加速并直接排出。
由于风扇的存在,涡扇发动机中的涡轮可分为高压涡轮和低压涡轮两个部分:高压涡轮负责带动压气机,低压涡轮则负责驱动风扇。
根据外涵道气流通过的路径,涡扇发动机可进一步分为分排式和混排式。
分排式发动机的外涵道气流不再和内涵道气流混合,而是通过外涵道尾喷管直接排出,这适用于外涵道较大(涵道比高)的情况,起飞推力大、巡航经济性好,多见于民航客机和运输机上。
混排式发动机的外涵道气流会经过混合器与内涵道气流重新混合,然后通过尾喷管,这适用于涵道比低的情况,设计紧凑、加力比大、亚声巡航经济性好,多见于超声速战机上。
桨扇发动机(Propfan)则可进一步视为涡扇发动机和涡桨发动机的结合,或是外涵道无限大的涡扇发动机,特点在于没有外涵道,而是直接使用裸露在外的螺旋桨。

变循环和组合循环发动机则是更先进的一类发动机。
其中变循环发动机可通过改变发动机的增压比、涡轮前温度、空气流量和涵道比等循环参数,在不同工作状况下使用不同的循环,从而优化各种情况下的发动机表现。
组合循环发动机则是将基于涡轮的布雷顿循环和其他类型的循环,例如冲压发动机的布雷顿循环,相组合形成的发动机。

\section{第二题}
我们现在主要考虑涡喷、涡桨和涡扇发动机这三种主流的航空发动机,也是课上主要研究的发动机,之间的结构差距以及影响它们结构的性能因素。

这三者之间结构的差距主要来自推力的来源,即推力主要来自排出气体的推力、螺旋桨还是两者的组合,影响这种结构差距的主要因素是推进效率与飞行速度的矛盾。
我们知道,涡喷发动机是热机和推进器的组合,其推进效率为
\begin{equation}
    \eta_p = \frac{(V_9 - V_0)V_0}{\frac{V_9^2 - V_0^2}{2}} = \frac{2}{1 + V_9 / V_0}.
\end{equation}
因此,当其低速飞行时,由于余速损失,发动机难以满足高推进效率和高热效率的要求。
另一方面,经典的螺旋桨结构在低速下效率较高,但难以突破音速,因此将两者相结合就是一种行之有效的思路。
这一性能因素导致的结构变化就是外涵道的出现以及涵道比的变化:
涡喷发动机没有外涵道,依赖纯喷气推进,高速($\mathrm{Ma}>2$)时动能转化效率高,但亚音速油耗极大。典型用于早期超音速战机。
涡扇发动机分内外涵道,通过外涵道低速气流与内涵道高速气流混合,在0.8-0.9马赫(民航巡航速度)实现最优推进效率,此外,涵道比越高,亚音速效率越高。
涡桨发动机使用的螺旋桨在低空低速($\mathrm{Ma}<0.6$)时推进效率超过$70\%$,但高速下桨尖激波导致效率骤降,故采用大直径螺旋桨与减速齿轮结构。

除了这一主要因素之外,还有几点其他的性能指标和考虑。
以推重比为例,对军用战斗机往往追求极高推重比,因此需要紧凑的结构设计,加之其需要较高的高速性能,因此主要选择涡轮喷气发动机(前期)以及小涵道比的涡扇发动机。
而对于民航客机使用的发动机,选择大涵道比涡扇发动机,不仅可以提高低速情况下的性能,还能以低速气流包裹高速喷流,从而降低混合噪声。
对于直升机,由于只需要轴功率而不需要喷气推进,因此只选择涡轮轴发动机。
