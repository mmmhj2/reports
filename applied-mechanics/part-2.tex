
\part{结构强度}

\begin{tcolorbox}
    有一实心均温等厚轮盘,轮缘半径为 \qty{0.25}{\meter};其材料质量密度为 \qty{8000}{kg/m^3},泊松比
为 0.3,许用应力为 \qty{800}{MPa}。转子工作转速为 \qty{12000}{r/min},且轮缘受有 \qty{200}{MPa} 的外载
荷,通过计算回答下述问题:
    \begin{enumerate}
        \item 该实心轮盘能否安全工作?画出轮盘的应力分布曲线,即半径(Y 坐标)-径向应
        力曲线(X 坐标)以及半径(Y 坐标)-周向应力(X 坐标)曲线。
        \item 若因总体结构设计方案改变,需要在盘心处开一中心孔,问轮盘能否安全工作?
        若不能,如何改进设计?
        \item 若该实心轮盘盘心温度为 200 度、盘缘温度为 500 度,且温度沿着轮盘径向线性
        变化,试计算轮盘上的热应力分布。画出热应力分布曲线,即半径(Y 坐标)-径向应力
        曲线(X 坐标)以及半径(Y 坐标)-周向应力(X 坐标)曲线。弹性模量及线膨胀系数
        可取常数:$E=\qty{160}{GPa}, \alpha=\qty{18e{-6}}{\degreeCelsius^{-1}}$。
        \item 计算该实心轮盘总应力,并判断轮盘能否安全工作。画出总应力分布曲线,即半
        径(Y 坐标)-径向应力曲线(X 坐标)以及半径(Y 坐标)-周向应力(X 坐标)曲线。
    \end{enumerate}
\end{tcolorbox}

\section{第一题}


