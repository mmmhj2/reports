
\part{结构强度}

\begin{tcolorbox}
    有一实心均温等厚轮盘,轮缘半径为 \qty{0.25}{\meter};其材料质量密度为 \qty{8000}{kg/m^3},泊松比
为 0.3,许用应力为 \qty{800}{MPa}。转子工作转速为 \qty{12000}{r/min},且轮缘受有 \qty{200}{MPa} 的外载
荷,通过计算回答下述问题:
    \begin{enumerate}
        \item 该实心轮盘能否安全工作?画出轮盘的应力分布曲线,即半径(Y 坐标)-径向应
        力曲线(X 坐标)以及半径(Y 坐标)-周向应力(X 坐标)曲线。
        \item 若因总体结构设计方案改变,需要在盘心处开一中心孔,问轮盘能否安全工作?
        若不能,如何改进设计?
        \item 若该实心轮盘盘心温度为 200 度、盘缘温度为 500 度,且温度沿着轮盘径向线性
        变化,试计算轮盘上的热应力分布。画出热应力分布曲线,即半径(Y 坐标)-径向应力
        曲线(X 坐标)以及半径(Y 坐标)-周向应力(X 坐标)曲线。弹性模量及线膨胀系数
        可取常数:$E=\qty{160}{GPa}, \alpha=\qty{18e{-6}}{\degreeCelsius^{-1}}$。
        \item 计算该实心轮盘总应力,并判断轮盘能否安全工作。画出总应力分布曲线,即半
        径(Y 坐标)-径向应力曲线(X 坐标)以及半径(Y 坐标)-周向应力(X 坐标)曲线。
    \end{enumerate}
\end{tcolorbox}

\section{第一题}

对于该问题,考虑使用位移法微分方程:
\begin{equation}
    \begin{aligned}
        \frac{\dif^2 u}{\dif r^2} + \left[ \frac{\dif}{\dif r} (\ln h )+ \frac{1}{r} \right] \frac{\dif u}{\dif r} + 
        \left[\frac{\nu}{r}\frac{\dif}{\dif r} (\ln h) - \frac{1}{r^2}\right] u & \\ 
        - (1 + \nu) \alpha \left[ t \frac{\dif}{\dif r} (\ln h) + \frac{\dif t}{\dif r}\right] +
        \rho \omega^2 \frac{1 - \nu^2}{E} r &= 0,
    \end{aligned}
\end{equation}
其中$r$为半径、$u$为位移、$h$为厚度、$t$为温度、$\omega$为角速度,$\nu, E, \rho, \alpha$分别为泊松比、杨氏模量、密度和线膨胀系数。
对等厚圆盘,有
\begin{equation}
    \frac{\dif h}{\dif r} = 0
\end{equation}
因此
\begin{equation}
    \begin{aligned}
        & \frac{\dif^2 u}{\dif r^2} + \frac{1}{r} \frac{\dif u}{\dif r} - \frac{u}{r^2} - (1 + \nu) \alpha \frac{\dif t}{\dif r} + \rho \omega^2 \frac{1 - \nu^2}{E} r = 0 \\
        \iff & \frac{\dif^2 u}{\dif r^2} + \frac{1}{r} \frac{\dif u}{\dif r} - \frac{u}{r^2} = (1 + \nu) \alpha \frac{\dif t}{\dif r} - \rho \omega^2 \frac{1 - \nu^2}{E} r \\
        \iff & \frac{\dif}{\dif r}\left( \frac{1}{r} \frac{\dif ru}{\dif r} \right) = (1 + \nu) \alpha \frac{\dif t}{\dif r} - \rho \omega^2 \frac{1 - \nu^2}{E} r
    \end{aligned}
\end{equation}
积分,可得
\begin{equation}
    \frac{1}{r} \frac{\dif ru}{\dif r} = (1 + \nu) \alpha t- \rho \omega^2 r^2 \frac{1 - \nu^2}{2E} + C_1'.
\end{equation}
再次积分,得到
\begin{equation}
    ru = (1 + \nu) \alpha \int_{r_0}^r tr \dif r - \frac{1}{8} \rho \omega^2 r^4 \frac{1 - \nu^2}{2E} + C_1 r^2 + C_2,
\end{equation}
其中$C_1', C_2$为积分常数,$C_1 = 2C_1'$.
根据本构方程(广义胡克定律),位移与应力之间具有以下关系:
\begin{equation}
    \begin{aligned}
        \epsilon_r = \frac{\dif u}{\dif r} &= \frac{\sigma_r - \nu \sigma_\theta}{E} + \alpha t, \\
        \epsilon_\theta = \frac{u}{r} &= \frac{\sigma_\theta - \nu \sigma_r}{E} + \alpha t.
    \end{aligned}
\end{equation}
代入并求解应力,可得
\begin{equation}
    \begin{aligned}
        \sigma_r &= K_1 - \frac{K_2}{r^2} -\frac{3+\nu}{8} \rho \omega^2 r^2 - \frac{\alpha E}{r^2} \int_{r_0}^r tr \dif r, \\
        \sigma_\theta & = K_1 + \frac{K_2}{r^2} -\frac{1+3\nu}{8} \rho \omega^2 r^2  - E \alpha t + \frac{\alpha E}{r^2} \int_{r_0}^r tr \dif r,
    \end{aligned}
\end{equation}
其中
\begin{equation}
    K_1 = \frac{C_1 E}{1-\nu}, K_2 = \frac{C_2 E}{1 + \nu},
\end{equation}
是两个根据边界条件求解的常数。
