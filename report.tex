\documentclass[12pt]{ctexart}
\usepackage{amsmath}
\usepackage[backend=biber,style=gb7714-2015]{biblatex}

\title{计算普朗特方程的数值解}
\author{***REMOVED*** \\ ***REMOVED***}

\addbibresource{references.bib}

\begin{document}
    \maketitle

\section{问题陈述}
我们知道,不可压缩的牛顿流体的流动满足纳维尔-斯托克斯方程:
\[
\left\{
\begin{aligned}
& \frac{\partial \vec U}{\partial t} + (\vec U \cdot \nabla) \vec U = - \frac{1}{\rho} \nabla p + \nu \nabla^2 \vec U + \vec f \\
& \nabla \cdot \vec U = 0
\end{aligned}
\right.
\]
该方程是一个复杂的非线性偏微分方程,其光滑解的存在性至今仍是未解之谜\cite{millenniumNSeq}。
为了更加方便地研究各种不同状态下流体的状态变化,流体力学家们一直在尝试简化该方程。

1909年,物理学家阿诺尔德·索末菲以奥斯鲍恩·雷诺的名字命名了一个新的常数,雷诺数,定义为\cite{sommerfeld1909beitrag}:
\[ \mathrm{Re} = \frac{Ul}{\nu} \]
雷诺数表征了流体的惯性作用和黏性作用之比\cite{10.1371/journal.pone.0141848},当其远大于1时,流体的惯性在流动中起主要作用,因此可以在一定程度上忽略黏性的作用,从而降低求解N-S方程的难度。

然而,N-S方程中的黏度项不仅与运动黏度$\nu$有关,还和流速的拉普拉斯算子有关,这意味着对速度梯度变化极快的区域,直接忽略黏性项会产生非常大的误差。
为了解决这一问题,流体力学家路德维希·普朗特提出了边界层的概念\cite{Tollmien1961}。
由于速度梯度的快速变化通常在壁面附近发生,他将高雷诺数的流体分为靠近壁面和远离壁面两个部分,并将靠近壁面的一部分称为边界层。
在边界层外,可以使用无黏性的近似来求解N-S方程;而在边界层内则依然需要考虑黏性。

普朗特利用量级分析将边界层中的N-S方程简化为以下方程:
\[
    \left\{
        \begin{aligned}
            & u \partial_x u + v \partial_y u = - \frac{1}{\rho} \partial_x p + \nu \partial_{yy} u \\
            & \partial_y p = 0 \\
            & \partial_x u + \partial_y v = 0
        \end{aligned}
    \right.
\]

我们考虑一侧为平板,另一侧为以固定速度流动的无黏性区域流体之间的边界层。
对无黏性的流体使用伯努利定律,可发现其压力不变,故$\partial_x p = 0$。
因此普朗特方程可进一步简化为:
\[
    \left\{
        \begin{aligned}
            & u \partial_x u + v \partial_y u = \nu \partial_{yy} u \\
            & \partial_x u + \partial_y v = 0
        \end{aligned}
    \right.
\]
我们将使用有限差分法求出该方程的数值解。

\section{有限差分法}

我们首先考虑动量方程,即第一个方程,因为该方程较为复杂。
注意到该方程是一个\emph{抛物偏微分方程},对$y$具有二阶偏导,而对$x$只具有一阶偏导,因此我们可以选择以$x$作为行进方向。
因此,我们将$u$沿$x$方向的导数写为后向差分,而将沿$y$方向的导数写成中间差分,该方程变为:
\[ 
    u_{x,y} \frac{u_{x+1, y} - u_{x,y}}{\Delta x} + v_{x,y} \frac{u_{x,y+1} - u_{x,y-1}}{2 \Delta y} = 
    \nu \frac{u_{x,y+1} - 2 u_{x, y} + u_{x,y-1}}{(\Delta y)^2}
\]
而质量连续性方程变为:
\[ \frac{u_{x+1,y} - u_{x,y}}{\Delta x} + \frac{v_{x+1,y} - v_{x+1,y-1}}{\Delta y} = 0 \]

因此,我们可以使用$u_{x,y+1}, u_{x, y}, u_{x,y-1}$来求解$u_{x+1,y}$。
注意到方程中还有一个未知数$v_{x,y}$,我们可以选择在计算$u$的同时利用连续性方程计算$v$,也可以选择先设$v$为某初始值,计算$u$,然后计算$v$,然后再迭代一次,重新计算。
第二种方法可称为预测-校正法,这种方法实现较为简单,因此我们选择这种方法。

\subsection{全显式法}

将第一个方程移项,可得:
\[ 
    u_{x+1,y} = \frac{\Delta x}{u_{x,y}} 
    \left( 
        \nu \frac{u_{x,y+1} - 2 u_{x, y} + u_{x,y-1}}{(\Delta y)^2}  
        - v_{x,y} \frac{u_{x,y+1} - u_{x, y-1}}{2 \Delta y} + \frac{u_{x,y} u_{x,y}}{\Delta x}
    \right) 
\]
正如上文所述,若需要求解$u_{x+1,y}$,我们需要先求出$u_{x,y+1}, u_{x, y}, u_{x,y-1}$,这意味着我们只能先沿$x$轴前进,再沿$y$轴前进。
只有这样才能保证在计算$u_{x+1, y}$时需要的值都已经求出。

同理,$v_{x+1,y}$也可以使用这种方法求解。

这种方式即为全显式法,其易于实现,不需要求解复杂的线性或非线性方程组,因此通常不需要进行迭代。

然而,其缺点在于稳定性不足。该方法的稳定性约束为\cite{book:9105}:
\[ \frac{2\nu \Delta x}{u_{x,y} (\Delta y)^2} \le 1 \quad \text{且} \quad \frac{v_{x,y}^2 \Delta x}{u_{x,y} \nu} \le 2 \]
可以注意到若网格在$y$方向上划分过细,则该算法反而会不稳定。
此外,该方法的稳定性也与求出的解有关,这意味着我们必须调整参数才能获得稳定的解。
这些问题促使我们寻找更优秀的算法。

\subsection{Du Fort-Frankel法}

为了解决全显式法的不稳定的问题,同时保持显式方法易于理解、编写和求解的优点,Du Fort和Frankel提出了一个更稳定的解决方案\cite{du1953stability}。
对于欲在$x$方向上使用步进方法求解的物理量$\phi_{x+1,y}$,可使用其在$x$方向上的平均来替代$\phi_{x, y}$,即在方程中将所有$\phi_{x,y}$代替为:
\[ \overline{\phi_{x,y}} = \frac{\phi_{x+1,y} + \phi_{x-1,y}}{2} \]

现在可以重写动量方程:
\begin{multline*}
    \frac{u_{x+1,y} + u_{x-1,y}}{2 \Delta x} \left( u_{x+1,y} - \frac{u_{x+1,y}+u_{x-1,y}}{2} \right) \\
    + v_{x,y} \frac{u_{x, y+1} - u_{x, y-1}}{2 \Delta y} \\
    = \frac{\nu}{(\Delta y)^2} (u_{x,y+1} - u_{x+1,y} - u_{x-1,y} + u_{x,y-1})
\end{multline*}
    

\subsection{精确解}

\section{结论}

\subsection{稳定性}

\subsection{准确性}

\printbibliography

\end{document}
