\documentclass[12pt]{ctexart}
\usepackage{amsmath}
\usepackage[backend=biber,style=gb7714-2015]{biblatex}

\title{计算普朗特方程的数值解}
\author{***REMOVED*** \\ ***REMOVED***}

\addbibresource{references.bib}

\begin{document}
    \maketitle

\section{问题陈述}
我们知道,不可压缩的牛顿流体的流动满足纳维尔-斯托克斯方程:
\[
\left\{
\begin{aligned}
& \frac{\partial \vec U}{\partial t} + (\vec U \cdot \nabla) \vec U = - \frac{1}{\rho} \nabla p + \nu \nabla^2 \vec U + \vec f \\
& \nabla \cdot \vec U = 0
\end{aligned}
\right.
\]
该方程是一个复杂的非线性偏微分方程,其光滑解的存在性至今仍是未解之谜\cite{millenniumNSeq}。
为了更加方便地研究各种不同状态下流体的状态变化,流体力学家们一直在尝试简化该方程。

1909年,物理学家阿诺尔德·索末菲以奥斯鲍恩·雷诺的名字命名了一个新的常数,雷诺数,定义为\cite{sommerfeld1909beitrag}:
\[ \mathrm{Re} = \frac{Ul}{\nu} \]
雷诺数表征了流体的惯性作用和黏性作用之比\cite{10.1371/journal.pone.0141848},当其远大于1时,流体的惯性在流动中起主要作用,因此可以在一定程度上忽略黏性的作用,从而降低求解N-S方程的难度。

然而,N-S方程中的黏度项不仅与运动黏度$\nu$有关,还和流速的拉普拉斯算子有关,这意味着对速度梯度变化极快的区域,直接忽略黏性项会产生非常大的误差。
为了解决这一问题,流体力学家路德维希·普朗特提出了边界层的概念\cite{Tollmien1961}。
由于速度梯度的快速变化通常在壁面附近发生,他将高雷诺数的流体分为靠近壁面和远离壁面两个部分,并将靠近壁面的一部分称为边界层。
在边界层外,可以使用无黏性的近似来求解N-S方程;而在边界层内则依然需要考虑黏性。

普朗特利用量级分析将边界层中的N-S方程简化为以下方程:
\[
    \left\{
        \begin{aligned}
            & u \partial_x u + v \partial_y u = - \frac{1}{\rho} \partial_x p + \nu \partial_{yy} u \\
            & \partial_y p = 0 \\
            & \partial_x u + \partial_y v = 0
        \end{aligned}
    \right.
\]

我们考虑一侧为平板,另一侧为以固定速度流动的无黏性区域流体之间的边界层。
对无黏性的流体使用伯努利定律,可发现其压力不变,故$\partial_x p = 0$。
因此普朗特方程可进一步简化为:
\[
    \left\{
        \begin{aligned}
            & u \partial_x u + v \partial_y u = \nu \partial_{yy} u \\
            & \partial_x u + \partial_y v = 0
        \end{aligned}
    \right.
\]
我们将使用有限差分法求出该方程的数值解。

\section{有限差分法}

我们首先考虑动量方程,即第一个方程,因为该方程较为复杂。
注意到该方程是一个\emph{抛物偏微分方程},对$y$具有二阶偏导,而对$x$只具有一阶偏导,因此我们可以选择以$x$作为行进方向。
因此,我们将$u$沿$x$方向的导数写为后向差分,而将沿$y$方向的导数写成中间差分,该方程变为:
\[ 
    u_{x,y} \frac{u_{x+1, y} - u_{x,y}}{\Delta x} + v_{x,y} \frac{u_{x,y+1} - u_{x,y-1}}{2 \Delta y} = 
    \nu \frac{u_{x,y+1} - 2 u_{x, y} + u_{x,y-1}}{(\Delta y)^2}
\]
而不可压缩方程变为:
\[ \frac{u_{x+1,y} - u_{x,y}}{\Delta x} + \frac{v_{x+1,y} - v_{x+1,y-1}}{\Delta y} = 0 \]

\subsection{全显式法}

\subsection{DuFort-Frankel法}

\section{解析解}

\printbibliography

\end{document}
