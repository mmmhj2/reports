% !TeX root = report.tex
\documentclass{ctexart}

\usepackage{tikz}
\usetikzlibrary{3d}

\title{三自由度机械臂建模分析}
\author{***REMOVED*** \\ ***REMOVED***}

\begin{document}

\maketitle

\section{问题简述}

随着工业自动化的大规模发展,机械臂在生产中得到了大规模的使用。
在众多机械臂中,三自由度机械臂凭借结构简单、控制容易以及造价低廉等优势受到青睐,
因此也成为重要的研究对象。

本技术报告将实现对简单的三自由度机械臂的运动学和动力学建模,并对其运动和控制在计算机中进行仿真。

\subsection{机械臂模型}

\begin{figure}[ht]
    \centering
    \begin{tikzpicture}[scale=1.5]
        % Axis
        \draw[->] (0,0,0) -- (xyz cylindrical cs:radius=5) node[below] {$y$};
        \draw[->] (0,0,0) -- (xyz cylindrical cs:radius=2,angle=90) node[right] {$z$};
        \draw[->] (0,0,0) -- (xyz cylindrical cs:z=2) node[right] {$x$};
        
        % Arm 1
        \draw[red, thick] (0,0,0) -- (0,1,0);
        
        % Arm 2
        \begin{scope}[blue]
            \draw[thick] (0,1,0) -- (2,2,1);
            \draw[dashed] (2,2,1) -- (2,0,1);
            \draw[dashed] (2,0,1) -- (0,0,1);
            \draw[dashed] (2,0,1) -- (2,0,0);
        \end{scope}
        
        % Arm 3
        \begin{scope}[green]
            \draw[thick] (2,2,1) -- (3,1,1.5);
            \begin{scope}[dashed]
                \draw (3,1,1.5) -- (3,0,1.5);
                \draw (3,0,1.5) -- (3,0,0);
                \draw (3,0,1.5) -- (0,0,1.5);
            \end{scope}
        \end{scope}
    \end{tikzpicture}
    \caption{机械臂示意图}
\end{figure}

\section{运动学模型}

\subsection{正向运动学}

\subsection{逆向运动学}

\section{动力学模型}

\end{document}
